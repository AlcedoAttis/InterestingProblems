\documentclass[12pt, a4paper]{article}
 
\usepackage[top=0.5in, bottom=0.9in, left=0.95in, right=0.95in]{geometry} 
\usepackage{amsmath,amsthm,amssymb,mathtools}
\usepackage{listings}
\usepackage{graphicx}
\usepackage{subcaption}
\usepackage{harpoon}

\newcommand{\vect}[1]{\overrightharp{\ensuremath{#1}}}
\newcommand{\field}[1]{\par\begin{large}{\vspace*{0.5cm}\noindent{}\textbf{#1}\vspace*{-1mm}}\end{large}}
\newcommand{\rom}[1]{\uppercase\expandafter{\romannumeral #1\relax}}
%\newcommand{\problem}[2]{{\par\noindent{}\textit{Problem {\uppercase\expandafter{\romannumeral #1\relax}}.} #2}}
\newenvironment{problem}[3]{{\par\vspace*{4.5mm}\noindent{}\textit{Problem {\uppercase\expandafter{\romannumeral #1\relax}}:} #2}\vspace*{-0.2cm}\hfill{\textit{{#3} Points.}}\\\hspace*{0.8cm}\hfill\begin{minipage}{\dimexpr\textwidth-1.3cm}}{\end{minipage}\hspace*{0.5cm}}
% don't fucking ask! (im 4h into this and it __works__)
 
\lstset{literate=
  {ä}{{\"a}}1 {ö}{{\"o}}1 {ü}{{\"u}}1
  {Ä}{{\"A}}1 {Ö}{{\"O}}1 {Ü}{{\"U}}1
  {ß}{{\ss}}1
}

\begin{document}
\title{Problem \rom{2} - Implicit Function}
\author{Philip Gei\ss{}ler \hspace{5cm} \textit{{4} Points}}
\date{}
\maketitle
\vspace*{-.9cm}

\begin{align*}
\varphi_\mathbb{Q} :&\mathbb{~~Q} \longmapsto \mathbb{Q} &\varphi_\mathbb{Q}(p)\cdot\varphi_\mathbb{Q}(q) &= \varphi_\mathbb{Q}(p+q)\\
\varphi_\mathbb{R} :&\mathbb{~~R} \longmapsto \mathbb{R} &\varphi_\mathbb{R}(p)\cdot\varphi_\mathbb{R}(q) &= \varphi_\mathbb{R}(p+q)
\end{align*}
~~~~~~~Which function do both functions $\varphi(x)$ represent? Are they continuous? 
\hrule
\vspace*{9mm}
Let's start with $\varphi_\mathbb{Q}(c), c\in\mathbb{Q}$:
\begin{align*}
\varphi(0) =&~ \varphi(0+0) = \varphi(0) \cdot \varphi(0) = \varphi(0)^2 \\
\Longrightarrow \varphi(0) =&~ \text{id}_{(\mathbb{R}, \cdot)} = 1\\
&~(\text{ or } \varphi(c) = 0 \Longrightarrow \varphi(q) = 0 ~~\forall b\in\mathbb{Q}\text{\,or\,}\forall b\in\mathbb{R})\\~\\
\varphi(c) =&~ \varphi\left({\frac{1}{n}\cdot c}\right) \cdot \varphi\left({\frac{n-1}{n}\cdot c}\right) & n\in \mathbb{N} \\
=&~\overbrace{\varphi\left({\frac{1}{n}\cdot c}\right) \cdot \varphi\left({\frac{1}{n}\cdot c}\right)\cdots \varphi\left({\frac{1}{n}\cdot c}\right)}^{n \text{ times}} = \varphi\left({\frac{c}{n}}\right)^n & n\in \mathbb{N}\\
\varphi(cm) =&~ \varphi\left({c}\right)^m \Longrightarrow \varphi(c) = \varphi\left({cm}\right)^\frac{1}{m} & m\in \mathbb{N}^+\\
\varphi\left(c\right) =&~ \varphi\left({\frac{m}{n}\cdot c}\right)^\frac{n}{m} & n\in \mathbb{N},m\in \mathbb{N}^+\\
\Longrightarrow \varphi\left(\frac{m}{n}\cdot c\right) =&~ \varphi\left({c}\right)^\frac{m}{n} = \varphi\left(q c\right) = \varphi\left({c}\right)^q & q\in \mathbb{Q}^+,~ n\in \mathbb{N},m\in \mathbb{N}^+\\~\\
\varphi(0) =&~ \varphi((q-q)c) = \varphi(qc) \cdot \varphi(-qc)\\
 =&~ \varphi(c)^q \cdot \varphi(-qc) = 1 \\
\Longrightarrow \varphi(-qc) =&~ \frac{1}{ \varphi(c)^q} =  \varphi(c)^{-q} & \Longrightarrow q\in \mathbb{Q}\\~\\
\varphi(1) \coloneqq&~ b & b \in \mathbb{?}\\
\varphi(q) =&~ \varphi(1)^q = b^q & \Longrightarrow b \in \mathbb{Q}^+ \text{ so } b^q \in \mathbb{Q} \\~\\
\Longrightarrow \varphi_\mathbb{Q}(q) =&~ b^q ~~\forall b\in\mathbb{Q}^+ ~\Longrightarrow  \varphi_\mathbb{Q}(q) \text{ is}\text{ continuous}\\
\Longrightarrow \varphi_\mathbb{R}(q) =&~ b^q ~~\forall q\in\mathbb{Q}~\forall b\in\mathbb{R}^+ & \text{analogous}
\end{align*}
So can we extend the definition of $\varphi_\mathbb{Q}$ to $\varphi_\mathbb{R}$? No!
\begin{align*}
\varphi_\mathbb{R}(q) &\coloneqq \left\{\begin{array}{rl}2^{y\pi}	 & \text{if } q = x + y\pi \mid x,y\in\mathbb{Q}\\ 1& \text{else}\end{array}\right.
\end{align*}
\end{document}

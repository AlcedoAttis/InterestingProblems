\documentclass[12pt, a4paper]{article}
 
\usepackage[top=0.5in, bottom=0.9in, left=0.95in, right=0.95in]{geometry} 
\usepackage{amsmath,amsthm,amssymb,mathtools}
\usepackage{listings}
\usepackage{graphicx}
\newcommand{\rom}[1]{\uppercase\expandafter{\romannumeral #1\relax}}
\pagenumbering{gobble}

\begin{document}
\title{Proof of the Rightfulness of $\varphi_p(x)$ as a Counterexample}
\author{}
\date{}
\maketitle
\vspace*{-1.9cm}
\centering
\begin{align*}
\varphi_p(q) \coloneqq \left\{\begin{array}{rl}2^{y} & \text{if } q = x + y\pi \mid x,y\in\mathbb{Q}\\ 0& \text{else}\end{array}\right. & & \varphi_p(q_1)\cdot\varphi_p(q_2) = \varphi_p(q_1+q_2)
\end{align*}\\\vspace*{7mm}
\begin{Large}Part \rom{1}: $\varphi_p(x)$ is a well-defined function\end{Large}
\begin{align*}
\text{to be shown:~~~~}&~\forall q_1,q_2\in \mathbb{R},~ x_1,y_1,x_2,y_2,x_s,y_s\in\mathbb{Q}\\
\left.\begin{array}{rl}2^{y} & \text{if } q_1+q_2 = x_s + y_s\pi\\ 0& \text{else}\end{array}\right\} =& \left\{\begin{array}{rl}2^{y} & \text{if } q_1 = x_1 + y_1\pi \\ 0& \text{else}\end{array}\right.\cdot\left\{\begin{array}{rl}2^{y} & \text{if } q_2 = x_2 + y_2\pi \\ 0& \text{else}\end{array}\right.\\
\varphi_p(q_1 + q_2)  =& ~\varphi_p(q_1) \cdot \varphi_p(q_2)\\
\end{align*}\vspace*{-15mm}
\begin{align*}
& & \mathllap{q_1,q_2\in \mathbb{R},~ x_1,y_1,x_2,y_2\in\mathbb{Q}}\\
\text{Case \rom{1}:~~} & q_1 = x_1 + y_1\pi ~\wedge~ q_2 = x_2 + y_2\pi & ~~\mathllap{i_1,i_2 \neq x_i+y_i\pi ~~\forall x_i,y_i\in\mathbb{Q}}\\
& q_1 + q_2 = (x_1+x_2) + (y_1+y_2)\pi\\
\Rightarrow~ & \varphi_p(q_1 + q_2) = 2^{y_1+y_2} = 2^{y_1} \cdot 2^{y_2} =\varphi_p(q_1) \cdot \varphi_p(q_2)\\
\text{Case \rom{2}:~~} & q_1 \neq x_1 + y_1\pi ~~\wedge~~ q_2 = x_2 + y_2\pi \vee q_1 = x_1 + y_1\pi ~\wedge~ q_2 \neq x_2 + y_2\pi\\
& q_1 + q_2 = x_a + y_a\pi + i_1\\
\Rightarrow~ & \varphi_p(q_1 + q_2) = 0 = 0 \cdot 2^{y_a} =\varphi_p(q_1) \cdot \varphi_p(q_2)\\
\text{Case \rom{3}:~~} & q_1 = x_1 + y_1\pi ~\wedge~ q_2 = x_2 + y_2\pi \\
& q_1 + q_2 = i_1 + i_2\\
\Rightarrow~ & \varphi_p(q_1 + q_2) = 0 = 0 \cdot 0 =\varphi_p(q_1) \cdot \varphi_p(q_2)\vspace*{20mm} \\
\Longrightarrow~ & \varphi_p(q_1 + q_2) = \varphi_p(q_1) \cdot \varphi_p(q_2) \phantom{a^{b^{c^{d^e}}}}\\
\end{align*}
\vspace*{0mm}\\
\begin{Large}Part \rom{2}: $\varphi_p(x)$ is not continuous\end{Large}
\begin{align*}
\text{continuity}&\Longleftrightarrow \forall x_0\in\mathbb{R} ~\forall \varepsilon>0 ~\exists\omega>0 ~\forall x\in \mathbb{R}    : |x-x_0|<\omega \Longrightarrow |\varphi_p(x)-\varphi_p(x_0)| < \varepsilon\\~\\
 & \hspace*{-2mm}\Longrightarrow~\exists x_0\in\mathbb{R} ~\exists \varepsilon>0 ~\nexists \omega>0 ~\forall x\in \mathbb{R} : |x-x_0|<\omega \Longrightarrow |\varphi_p(x)-\varphi_p(x_0)| < \varepsilon\\
& \Leftrightarrow  ~\exists x_0\in\mathbb{R} ~\exists \varepsilon>0 ~\forall \omega>0 ~\exists x\in \mathbb{R} : |x-x_0|<\omega \Longrightarrow |\varphi_p(x)-\varphi_p(x_0)| \geqslant \varepsilon\\
& \Leftrightarrow ~\text{noncontinuity}\\~\\
&\hspace*{-0.5cm} \varepsilon_0 \coloneqq 0.17491 & \mathllap{\text{(because }0.17491<1)}\\[-3mm]
&\hspace*{-0.5cm} x_0 \coloneqq \pi, ~~x \coloneqq \lim_{n\rightarrow\infty}\frac{\lfloor{24^n\pi}\rfloor}{\phantom{\lceil}{24^n}\phantom{\pi\rceil}} = q \in \mathbb{Q}, ~~\forall\omega>0: |x-\pi|<\omega & \mathllap{\text{(because }24\gg1)}\\
&\hspace*{-0.5cm} \varphi_p(\pi)=2, ~~\varphi_p(q)=1 \hspace{17.5mm}\Longrightarrow \forall\omega>0: |\varphi_p(x)-\varphi_p(x_0)| = 1 > \varepsilon_0
\end{align*}\\~\\~\\
\begin{Large}$\Longrightarrow \varphi_p(x)$ is a noncontinuous function\end{Large}
\end{document}



%\varphi_\mathbb{Q} :&\mathbb{~~Q} \longmapsto \mathbb{Q} &\varphi_\mathbb{Q}(p)\cdot\varphi_\mathbb{Q}(q) &= \varphi_\mathbb{Q}(p+q)\\
%\varphi_\mathbb{R} :&\mathbb{~~R} \longmapsto \mathbb{R} &\varphi_\mathbb{R}(p)\cdot\varphi_\mathbb{R}(q) &= \varphi_\mathbb{R}(p+q)

% Fachkonferenz Informatik des CZG
% Handout zum Vortrag im Rahmen des Spezi-Abiturs
% -----------------------------------------------------------------------
% Präambel (head)
\documentclass[a4paper,12pt]{scrreprt}

% Standardpakete
\usepackage[utf8]{inputenc}
\usepackage[english]{babel}
\usepackage[T1]{fontenc}
\usepackage{lmodern}

\usepackage{pgfplots}

% Zusatzpakete
\usepackage[affil-it]{authblk}
\usepackage{graphicx}
\usepackage{tabularx}

% Definitionen und Beispiele ohne Nummerierungen einbinden
\usepackage{ntheorem}
\newtheorem*{defi}{Definition}
\newtheorem*{bsp}{Examples}

% Darstellung von URLs
\usepackage{hyperref}

% seitliche Abbildungsbeschriftung
\usepackage{sidecap}

% Bilder und Grafiken nebeneinander
\usepackage{caption}
\usepackage{subcaption}

% Nummerierung der Abbildungen
\usepackage{tocloft}
\renewcommand{\cftfigpresnum}{Abb. }
\renewcommand{\thefigure}{\arabic{figure}}
\renewcommand{\cftfigaftersnum}{:}
\setlength{\cftfignumwidth}{2cm}
\setlength{\cftfigindent}{0cm}
\setlength{\textheight}{1.1\textheight}
% Einrücktiefe 0
\parindent=0pt

% Titelseite
% Logo in die Titelseite einbinden
\titlehead{
	\centering
	\includegraphics[width=3cm]{czg.png}% 
}
\subject{Computer Science Presentation}
\title{$\mu$ - recursive functions}
\author{Philip Geissler}
\affil{Carl-Zeiss Grammar School}
% Vortragsdatum:
% \today setzt das Datum des letzten Übersetzens
% statt \today kann ein konkretes Datum gesetzt werden
\date{Jena, 17th of November}
\publishers{Kai Rodeck}
%-----------------------------------------------------------------------

% Text (body)
\begin{document}

% Titelseite einbinden	
\maketitle	
% -----------------------------------------------------------------------

% zweite Seite
% Gliederung - nur Hauptpunkte des Vortrags aufführen
\section*{Structure}
\begin{enumerate}
	\item Computability
	\item $\mu$ - recursive functions
	\begin{itemize}
		\item Explanation of terms
		\item Initial functions
		\item recursion / composition / $\mu$-operator
	\end{itemize}
	\item Examples for primitive and $\mu$ recursion functions
\end{enumerate}

\section*{Theory of Computability}
\subsection*{Definition of computability}
A function $f(x_1,x_2,\cdots,x_n)$ is computable if there exists an algoritm $A(x_1,x_2,\cdots,x_n)$ following given rules, that solves f in its domain and gives no output otherwise.\\ \\
$(x_1,x_2,\cdots,x_n) \in D_f \Longrightarrow  A(x_1,x_2,\cdots,x_n) = f(x_1,x_2,\cdots,x_n)$\\
$(x_1,x_2,\cdots,x_n) \notin D_f \Longrightarrow  A(x_1,x_2,\cdots,x_n) = \textbf{ENDLESS LOOP}$\\

\subsection*{Theories of computability}
\begin{itemize}
\item{turing machines}
\item LOOP-programs (incomplete)
\item{GOTO-programs}
\item{WHILE-programs}
\item{$\lambda$-calculus}
\item{} primitive recursive functions (incomplete)
\item{} $\mu$-recursive functions
\end{itemize}

% Hier die Grafik einfügen!!!!
\begin{figure}[h]
\begin{tikzpicture}[scale=1.3]
\node (A) at (-2,1){$\mu$ - recursive functions};
\node (B) at (2,1){primitive recursive functions};
\node (C) at (2,0){LOOP-programs};
\node (D) at (-2,0){WHILE-programs};
\node (E) at (-4,-1){GOTO-programs};
\node (F) at (-1,-1){Turing machines};

\draw[<-] (A) edge (B);
\draw[<->] (A) edge (D);
\draw[<->] (C) edge (B);
\draw[->] (C) edge (D);
\draw[<->] (D) edge[bend right=30] (E);
\draw[<->] (F) edge[bend right=30] (D);
\draw[<->] (E) edge[bend right=30] (F);

\end{tikzpicture}
\end{figure}

% Seitenumbruch
\newpage
% -----------------------------------------------------------------------
% zweite Seite
\section*{Building-Block functions}
\subsection*{Constant function}
$N(X_0,x_1,x_2,\cdots,x_n)$:\\
$N(x_1,x_2,\cdots,x_n) = const. $
\subsection*{Successor function}
$S(x_1)$:\\
$S(x_1) = x_1 + 1$
\subsection*{Projection function}
$P_m^n(x_1,x_2,\cdots,x_n)$:\\
$P_m^n(x_1,x_2,\cdots,x_m,\cdots,x_n) = x_m$
\subsection*{Composition function}
$C(x_1,x_2,\cdots,x_n)$:\\
$f_n = f_n(x_1,x_2,\cdots,x_n) $\\
$f_{m,k} = f_{m,k}(x_1,x_2,\cdots,x_m) $\\
$C = f_n \circ (f_{m,1},f_{m,2},\cdots,f_{m,n})$\\
$C= f_n(f_{m,1}(x_{1,1},x_{1,2},\cdots,x_{1,m}),\cdots,f_{m,n}(x_{n,1},x_{n,2},\cdots,x_{n,m}))$
\subsection*{Primitive recursion function}
$R(X_0,x_1,x_2,\cdots,x_n)$:\\
$X_0=0 \Longrightarrow ~ R(X_0,x_1,x_2,\cdots,x_n) = f(y_1,\cdots,y_m)$\\
$X_0\neq0 \Longrightarrow ~ R(X_0,x_1,x_2,\cdots,x_n) = g(R(X_0-1,x_1,\cdots,x_n),z_1,\cdots,z_l)$
\subsection*{$\mu$ minimisation function}
$\mu(X_0,x_1,x_2,\cdots,x_n)$:\\
$0 = f(z,y_1,y_2,\cdots,y_n) $\\
$\wedge~ 0< f(i,y_1,y_2,\cdots,y_n)$\\
$\Longleftrightarrow z = \mu(f(Y_0,y_1,y_2,\cdots,y_n),x_1,x_2,\cdots,x_n)$
%------------------------------------------------------------------------

\newpage
	\begin{thebibliography}{xxxx}
		\bibitem[Access Date]{}
			\today
		\bibitem[Picture]{}
			URL: \url{https://goo.gl/gRrDGH}
		\bibitem[Overall]{}
			URL: \url{https://goo.gl/yWvinS}
		\bibitem[Overall]{}
			URL: \url{https://goo.gl/awUSfB}
		\bibitem[Overall]{}
			URL: \url{https://goo.gl/xRPCiU}
		\bibitem[Overall]{}
			URL: \url{https://goo.gl/K4LKQB}
		\bibitem[Overall]{}
			URL: \url{https://goo.gl/XudPdL}
		\bibitem[Overall]{}
			U.Schöning, Theoretische Informatik kurz gefasst, 2008, Spektrum

			
	
\end{thebibliography}
% -----------------------------------------------------------------------
\end{document}


%%% http://marslanalam.blogspot.de/2012/08/jpeg-mpeg-compression.html
%%7
%%
%%gtrrtnj8rhugzhubhzg
%%dhjufkbwgtngebtkg
%zghuiugzhujokitrhuijotk
%fghih57uji6
%
%
%
'

\documentclass[a4paper]{scrartcl} 
\usepackage[ngerman]{babel} 
\usepackage{pgfplots}
\usepackage{amsmath,amsthm,amssymb}
\usepackage[T1]{fontenc}
\usepackage[utf8]{inputenc}
\usepackage{listings}
\usepackage{relsize}
\usepackage[top=.25in, bottom=0.5in, left=.5in, right=0.5in]{geometry} %nice

%\usepackage{showframe}

\pgfplotsset{compat=1.15}
\usetikzlibrary{datavisualization}

\lstset{literate=
  {ä}{{\"a}}1 {ö}{{\"o}}1 {ü}{{\"u}}1
  {Ä}{{\"A}}1 {Ö}{{\"O}}1 {Ü}{{\"U}}1
  {ß}{{\ss}}1
}

\newcommand*\dif{\mathop{}\!\mathrm{d}} 

% function ill use 
% 2^y * x^(-(2^y+1))
% pow(2,y) * pow(x,(-(pow(2,y)+1)))
%
%
%

\definecolor{at0}{rgb}{.8,0,0}
\definecolor{at1}{rgb}{.4,0,0}
\definecolor{at2}{rgb}{.2,0,0}
\definecolor{at3}{rgb}{.1,0,0}
\definecolor{f1}{rgb}{.8,1,0}
\definecolor{f2}{rgb}{.8,1,.8}
\pgfkeys{/pgfplots/legend entry/.code=\addlegendentry{#1}}

\begin{document}
\title{\huge\normalfont\bfseries\text{Konvergente Statistische Folgen	}}
\author{}
\date{}
\maketitle

%\begin{figure}[h]
\begin{center}
\vspace*{-2.3cm}
{\Large Philip Geißler}\\
\begin{tikzpicture}
\begin{axis}[
axis x line=center,
  axis y line=center,
  xtick={1.001,1.75,2,3,4},
  xticklabels={1,$\varepsilon$,2,3,4},
  ytick={0.001,0.5,1,1.5},
  yticklabels={0,$\frac{1}{2}$,1,$\frac{3}{2}$},
  xlabel={$x$},
  ylabel={$p(x)$},
  xlabel style={below right},
  ylabel style={above left},
  xmin=1,
  xmax=5,
  ymin=0,
  ymax=1.45,
  scale only axis,
  width = 0.87\linewidth,
  height = 0.25\linewidth ]

\addplot[domain=1.74:1.76 ,samples=5, smooth, white!80!black, line width = .5pt, forget plot, dashed]
{1000*x-1750};
\addplot[domain=1:1.75 ,samples=50, smooth, fill=f1!30!white,forget plot] {pow(2,0) * pow(x,(-(pow(2,0)+1)))} \closedcycle;
\addplot[domain=1.75:4.97 ,samples=50, smooth, fill=f2!50!white,forget plot] {pow(2,0) * pow(x,(-(pow(2,0)+1)))} \closedcycle;
\addplot[domain=1:4.97 ,samples=500, smooth, white!80!at3, legend entry=$p_{5}(x)$, line width = .5pt]
{pow(2,5) * pow(x,(-(pow(2,5)+1)))};
\addplot[domain=1:4.97 ,samples=50, smooth, white!40!at2, legend entry=$p_{2}(x)$, line width = .66pt]
{pow(2,2) * pow(x,(-(pow(2,2)+1)))};
\addplot[domain=1:4.97 ,samples=50, smooth, white!20!at1, legend entry=$p_{1}(x)$, line width = 1pt]
{pow(2,1) * pow(x,(-(pow(2,1)+1)))};
\addplot[domain=1:4.97 ,samples=50, smooth, white!10!at0, legend entry=$p_{0}(x)$, line width = 1.5pt]
{pow(2,0) * pow(x,(-(pow(2,0)+1)))};

\addplot[domain=-2:-1 ,samples=5, smooth, f1!30!white, line width = 10	pt, legend entry=$P^{[\varepsilon]}_{\neg{A}}(0)$]
{x};
\addplot[domain=-2:-1 ,samples=5, smooth, f2!50!white, line width = 10	pt, legend entry=$P^{[\varepsilon]}_{{A}}(0)$]
{x};
\end{axis}
\end{tikzpicture}
%\end{figure}
\end{center}
\centering
\begin{minipage}[b]{0.84\linewidth}
Man definiere sich eine Wahrscheinlichkeitsverteilungsfunktion $p_{n}(x)$, die über einen Parameter $n$ verändert werden kann und auf ihrem Definitionsbereich eine Einheitsfläche einschließe:
\end{minipage}
\begin{align*}
p_{n}(x) &= \frac{2^n}{x^{2^n+1}} & \int_{1}^{\infty} p_{n}(x) \dif x&= \left[ -\frac{1}{x^{2^n}} \right]_1^{\infty} = 1 & \text{DB} \left(p_{n}(x)\right) &= \left[ 1,\infty\right[\\
\end{align*}
\begin{minipage}[t]{0.84\linewidth}
\vspace*{-0.8cm}
Darauffolgend definiere man durch Zweiteilung des Definitionsbereich von $p_{n}(x)$ zwei Ereigniswahrscheinlichkeiten $P^{[q]}_A(n)$ und $P^{[q]}_{\neg{A}}(n)$ dafür, dass $x$ in, oder eben nicht in $[q,\infty[$ liegt. Außerdem bilde man eine allgemeine Bereichswahrscheinlichkeit $P^{[p,q]}(n)$:
\end{minipage}
\begin{align*}
P^{[q]}_A(n) &= \int_{q}^{\infty} p_{n}(x) \dif x = q^{-(2^n)} & P^{[p,q]}(n) &= \int_{p}^{q} p_{n}(x) \dif x\\
\Rightarrow P^{[q]}_{\neg{A}}(n) &= \int_{1}^{q} p_{n}(x) \dif x = 1-q^{-(2^n)}\\
\end{align*}	
\begin{minipage}[t]{0.84\linewidth}
\vspace*{-0.8cm}
Zusätzlich bilde man $Q_{\neg{A}}^{[q]}(k)$, welches die Wahrscheinlichkeit für dauerhaftes Auftreten von $\neg A$ und damit für das Nichtauftreten von $A$ in $p_k(x),\hspace{1pt} p_{k+1}(x),\hspace{1pt} p_{k+2}(x),\hspace{1pt} \ldots$ angibt:
\end{minipage}
\begin{align*}
\Rightarrow \textstyle{Q}_{\neg{A}}^{[q]}(k) &= p({\neg A}_{k}\wedge{\neg A}_{k+1}\wedge\ldots) = \prod_{i=k}^{\infty} P^{[q]}_{\neg{A}}(i) = \prod_{i=k}^\infty 1-q^{-(2^i)} >0 &:{q>1	}\\
 \lim_{k \rightarrow \infty} {Q}_{\neg{A}}^{[q]}(k)&=1
\end{align*}
\begin{minipage}[t]{0.84\linewidth}
%\vspace*{-0.8cm}
Nun kann eine statistische Folge $(x_n)_{n=0}^{\infty}$ auf Basis des Wahrscheinlichkeitsmaßes erstellt werden, deren Glieder eine (unterschiedliche) Wahrscheinlichkeitsverteilung für ihre Werte besitzen:
\end{minipage}
\vspace*{-0.3cm}
\begin{align*}
\text{Folge: }& (x_n)_{n=0}^{\infty} \hspace*{2cm} x_n \in \text{ DB}(P_{n}(x)) : p(x_n\in [b,d]) = P^{[b,d]}(n)\\
\text{klassische Konvergenz: }& \forall\varepsilon > 0, ~\exists n_0 \in \mathbb{N}, ~\forall n \geqslant n_0 : | x_n - a| < \varepsilon
\end{align*}
\begin{minipage}[t]{0.84\linewidth}
Um schließlich das Konvergenzkiterium sinnvoll auf statistische Folgen zu erweitern, muss man noch eine beliebig kleine Wahrscheinlichkeit $\omega_n = 1- \textstyle{Q}_{\neg{A}}^{[\varepsilon]}(n)$, $(n\geqslant n_0)$ für das zukünftige Auftreten eines Gliedwertes außerhalb der $\varepsilon$-Umgebung verlangen können:
\end{minipage}
\begin{align*}
\text{statistische Konvergenz:  }
\forall\varepsilon,\omega > 0, ~\exists n_0 \in \mathbb{N}, ~\forall n \geqslant n_0 : p\left(| x_n - 1| > \varepsilon\right) = 1- \textstyle{Q}_{\neg{A}}^{[\varepsilon]}(n) < \omega
\end{align*}
\begin{minipage}[t]{0.84\linewidth}
Unsere Folge $(x_n)_{n=0}^{\infty}$ erfüllt dies und ist damit trotz der Möglichkeit beliebig großer Werte von beliebig späten Gliedern statistisch konvergent. \hfill{} \textit{q.e.d.}	
\end{minipage}
\end{document}


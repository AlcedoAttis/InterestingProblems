\documentclass[12pt]{report} 
\usepackage[margin=1in]{geometry} 
\usepackage{amsmath,amsthm,amssymb}
\usepackage{graphicx}
\usepackage[utf8]{inputenc}
\usepackage{titlesec}
\def\arraystretch{1.5}    %nicht
\setlength{\parskip}{0pt} %kaputtmachen	
\begin{document}
 
% --------------------------------------------------------------
%                         Start here
% --------------------------------------------------------------
  
\title{\begin{Huge}\textbf{Quantencomputer etc Titel...}\end{Huge}\\		% Titel
Nutzen für Menschen etc Untertitel}											% Untertitel
\author{Philip Geißler, Alexander von Mach, Joe Schaller\\					% Authoren
Stephan Fritzsche?, Vorname Hertenberger?, Ilona Böttcher?}					% Coauthoren, Hilfe ?? (optional) 
\maketitle
% Dieser Teil ist nur für Joe. Das ist die Überschrift!


%LAYOUTIDEE
% bitte nicht in direkt in ein Capter schreiben!
% Statdessen Sections darunter öffnen und in diese schreiben
\chapter{Einführung Quantencomputer}
\section{•}
bla
\section{•}
blah
\chapter{Qubits}							% X
\section{Was sind Qubits}					% X.Y
blubb
\subsection{Theoretische Qubitarten}		% X.Y.Z
blibb



\section{Erster Teil}
Bsp: Mathematik\\
Notiz 1: Das Sternchen bei "align" sorgt dafür das die einzelnen Formeln nicht nummmeriert werden.\\
Notiz 2: Das Dollarzeichen nicht im grünen Teil benutzen!
Notiz 3: proof macht so ein Quadrat ans Ende.
Notiz 4: Wenn ihr was nacheinander macht immer ein \& (ohne Backslash) dazwischen, sonst geht was kaputt.
\begin{proof}[Name des Beweises]\begin{align*}
\sum_{i=1}^{k+1}i & = \left(\sum_{i=1}^{k}i\right) +(k+1)\\ 
& = \frac{(k+1)((k+1)+1)}{2}\\
\mathbb{N} & \text{ ist das Zeichen für natürliche Zahlen.}
\end{align*}\end{proof}

\begin{align*}
Matrixa &=
\begin{pmatrix}
	A&B&C&D\\
	a&b&c&d\\
	a&B6c&D\\
	A&b&C&d
\end{pmatrix}
\end{align*}

\section{Zweiter Teil}
Bsp: Text\\
Überraschung. Text ist Text\\
\begin{Huge} Text kann riesig \end{Huge}\\
\begin{tiny} bis mini sein \end{tiny}\\
Nachvollziebar nutzen!

\newpage	% Formatiert das folgende auf die nächste Seite.
\section{Dritter Teil}
Bsp: Tabellen\\
Notiz 1: Für größere Tabellen größere Schrift verwenden.
Notiz 2: Das cl in den Klammern sorgt für das Layout: 
l=left, c=center, r=right\\\\

\begin{tabular}[]{clr}
	A 			& B 		& ... \\
	\hline
	erstens    	& qwerzt	& Nichts\\
	zweitens   	& asdfgh	&       \\
	drittens   	& yxcvbn
\end{tabular}

\section{Vierter Teil}
Bsp: Bilder\\
Notiz 1: Breite A4: 21,95cm, Höhe A4: 35,56cm\\
%\includegraphics[width=10cm]{bildname.png}

\end{document}
% Falls Fragen auftreten, dann bitte mich FÜR eine Lösung, oder mich MIT einer Lösung anschreiben,
% damit wir die gleichen Layouts nutzen und nicht jeder etwas abändert.